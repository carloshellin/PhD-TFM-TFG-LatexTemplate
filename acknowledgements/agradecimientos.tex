%%%%%%%%%%%%%%%%%%%%%%%%%%%%%%%%%%%%%%%%%%%%%%%%%%%%%%%%%%%%%%%%%%%%%%%%%%%
%
% Generic template for TFC/TFM/TFG/Tesis
%
% $Id: agradecimientos.tex,v 1.7 2015/06/05 00:10:31 macias Exp $
%
% By:
%  + Javier Macías-Guarasa. 
%    Departamento de Electrónica
%    Universidad de Alcalá
%  + Roberto Barra-Chicote. 
%    Departamento de Ingeniería Electrónica
%    Universidad Politécnica de Madrid   
% 
% Based on original sources by Roberto Barra, Manuel Ocaña, Jesús Nuevo,
% Pedro Revenga, Fernando Herránz and Noelia Hernández. Thanks a lot to
% all of them, and to the many anonymous contributors found (thanks to
% google) that provided help in setting all this up.
%
% See also the additionalContributors.txt file to check the name of
% additional contributors to this work.
%
% If you think you can add pieces of relevant/useful examples,
% improvements, please contact us at (macias@depeca.uah.es)
%
% You can freely use this template and please contribute with
% comments or suggestions!!!
%
%%%%%%%%%%%%%%%%%%%%%%%%%%%%%%%%%%%%%%%%%%%%%%%%%%%%%%%%%%%%%%%%%%%%%%%%%%%

% Use this if you don't like the fancy style
\thispagestyle{empty}

\ifthenelse{\equal{\myLanguage}{english}}
{
  \chapter*{Acknowledgements}
  \label{cha:acknowledgements}
  \markboth{Acknowledgements}{Acknowledgements}
  \addcontentsline{toc}{chapter}{Acknowledgements}
}
{
  \chapter*{Agradecimientos}
  \label{cha:agradecimientos}
  \markboth{Agradecimientos}{Agradecimientos}
  \addcontentsline{toc}{chapter}{Agradecimientos}
}




\begin{FraseCelebre}
  \begin{Frase}
    A todos los que la presente vieren y entendieren.
  \end{Frase}
  \begin{Fuente}
    Inicio de las Leyes Orgánicas. Juan Carlos I
  \end{Fuente}
\end{FraseCelebre}

% ``Más vale un minuto de ilusión que mil horas de
% razonamiento''... (cortesía de Roberto Barra)


Este trabajo es el fruto de muchas horas de trabajo, tanto de los
autores últimos de los ficheros de la distribución como de todos los que
en mayor o menor medida han participado en él a lo largo de su proceso
de gestación.

Mención especial merece Manuel Ocaña, el autor de la primera versión de
las plantillas de proyectos fin de carrera y tesis doctorales usadas en
el Departamento de Electrónica de la Universidad de Alcalá, con
contribuciones de Jesús Nuevo, Pedro Revenga, Fernando Herránz y Noelia
Hernández.

En la versión actual, la mayor parte de las definiciones de estilos de
partida proceden de la tesis doctoral de Roberto Barra-Chicote, con lo
que gracias muy especiales para él.

Añado igualmente a Gonzalo Corral de forma destacada en esta sección por ser el
primero que hizo un PR sobre el repositorio github de la plantilla, y que nos
llevó del arcaico ISO-8859 de bibtex a la normalidad del UTF-8 de biber (que ya
hacía falta).

También damos las gracias a \input{additionalContributors.txt} que nos
han proporcionado secciones completas y ejemplos puntuales de sus
proyectos fin de carrera.

Finalmente, hay incontables contribuyentes a esta plantilla, la mayoría
encontrados gracias a la magia del buscador de Google. Hemos intentado
referenciar los más importantes en los fuentes de la plantilla, aunque
seguro que hemos omitido alguno. Desde aquí les damos las gracias a
todos ellos por compartir su saber con el mundo.


% Back to normal JIC. Use it if you set \pagestyle{myplain} above
%\pagestyle{fancy}

%%% Local Variables:
%%% TeX-master: "../book"
%%% End:


