%%%%%%%%%%%%%%%%%%%%%%%%%%%%%%%%%%%%%%%%%%%%%%%%%%%%%%%%%%%%%%%%%%%%%%%%%%% 
% 
% Generic template for TFC/TFM/TFG/Tesis
% 
% $Id: introduccion.tex,v 1.22 2020/03/24 17:18:20 macias Exp $
% 
% By:
% + Javier Macías-Guarasa. 
% Departamento de Electrónica
% Universidad de Alcalá
% + Roberto Barra-Chicote. 
% Departamento de Ingeniería Electrónica
% Universidad Politécnica de Madrid   
% 
% Based on original sources by Roberto Barra, Manuel Ocaña, Jesús Nuevo,
% Pedro Revenga, Fernando Herránz and Noelia Hernández. Thanks a lot to
% all of them, and to the many anonymous contributors found (thanks to
% google) that provided help in setting all this up.
% 
% See also the additionalContributors.txt file to check the name of
% additional contributors to this work.
% 
% If you think you can add pieces of relevant/useful examples,
% improvements, please contact us at (macias@depeca.uah.es)
% 
% You can freely use this template and please contribute with
% comments or suggestions!!!
% 
%%%%%%%%%%%%%%%%%%%%%%%%%%%%%%%%%%%%%%%%%%%%%%%%%%%%%%%%%%%%%%%%%%%%%%%%%%% 

\chapter{Introducción}
\label{cha:introduccion}


\section{Introducción}
\label{sec:introduccion}

\subsection{Definición del problema}
\label{sec:definicion-problema}

\subsection{Preguntas de investigación}
\label{sec:preguntas-investigacion}

%\section{Objetivos de investigación}
%\label{sec:objetivos-investigacion}

\section{Estado del arte}
\label{sec:estado-arte}

\section{Contribución}
\label{sec:contribucion}

\section{Resumen de la contribución}
\label{sec:resumen-contribucion}

\section{Estructura}
\label{sec:estructura}







%%% Local Variables:
%%% TeX-master: "../book"
%%% End:


